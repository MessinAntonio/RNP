\section{Il nucleo atomico: nozioni introduttive e terminologia}
Un atomo è costituito da un nucleo centrale di carica elettrica positiva e da una nube elettronica di carica elettrica negativa, legata al nucleo mediante la \textit{forza Coulombiana}. Il più semplice atomo conosciuto è l'atomo di idrogeno H e consiste di un \textbf{elettrone} (e) e di un \textbf{protone} (p), che forma il nucleo. Tutti gli altri nuclei sono sistemi legati, composti da una combinazione di $N$ \textbf{neutroni} (n) e $Z$ \textit{protoni}. Data la neutralità degli atomi, il numero $Z$ corrisponde anche al numero di elettroni. Tale numero, che viene chiamato \textbf{numero atomico} o \textit{numero protonico}, è pari al numero di protoni presenti nel nucleo ed identifica univocamente un elemento della tavola periodica. Ad esempio con $Z = 8$ identifichiamo l'ossigeno O oppure con $Z = 26$ il ferro Fe. I \textit{neutroni} e i \textit{protoni} sono anche chiamati \textbf{nucleoni}, essendo le particelle costituenti del nucleo atomico.
    \begin{table}[ht]
        \centering
        \begin{tabular}{ccccc}
            \toprule
            Particella & Simbolo & Carica & Massa & Spin \\
            \midrule
            Elettrone & e & $-e$ & $9.109 \times 10^{-31} \, \mathrm{kg}$ & $1/2$ \\
            Protone & p & $e$ & $1.673 \times 10^{-27} \, \mathrm{kg}$ &  $1/2$ \\
            Neutrone & n & $0$ & $1.675 \times 10^{-27} \, \mathrm{kg}$ &  $1/2$ \\
            \bottomrule
        \end{tabular}
        \caption{Carica elettrica, massa e spin dell'elettrone e dei nucleoni.}
        \label{tab: atom's particles}
    \end{table}
\\La carica elementare $e$ è stata fissata esattamente dal sistema internazionale nel 2019 ed è pari a 
    \begin{equation*}
        e = 1.602 \, 176 \, 634 \times 10^{-19} \, \mathrm{C}
    \end{equation*}
Poiché il protone e il neutrone hanno una massa tre ordini di grandezza maggiore di quella dell'elettrone, la quasi totalità della massa atomica è concentrata nel nucleo. La somma $Z + N$, ovvero il numero di nucleoni presenti nel nucleo, prende il nome di \textbf{numero di massa} ed è indicata con $A$. Per indicare una determinata specie nucleare o \textbf{nuclide} sarà utilizzata la seguente notazione
    \begin{equation*}
        {}^A_Z X_N
    \end{equation*}
dove con $X$ è indicato il simbolo chimico. Per gli elementi esistenti in natura, $Z$ varia tra 1 (idrogeno) e 92 (uranio), mentre $A$ assume valori da 1 fino ad un massimo di 238. È evidente come tale notazione risulti ridondante dato che insieme al simbolo chimico, che individua univocamente il numero atomico $Z$, basterebbe soltanto il numero di neutroni $N$ o il numero di massa $A$.
    \begin{equation*}
        {}^A_Z X_N \equiv {}^{A}X 
    \end{equation*}
In questo modo ${}^{238} \text{U}$ è un modo perfettamente valido per indicare il nuclide uranio-238 ${}^{238}_{92} \text{U}_{146}$. Per comprendere la struttura e le proprietà degli atomi e dei nuclei atomici diamo ora una serie di definizioni:
    \begin{itemize}
        \item \textbf{Isotopi}: Due nuclidi con uguale $Z$ e diverso $A$. Gli \textit{isotopi} sono dunque atomi dello stesso elemento chimico che hanno lo stesso numero atomico $Z$, ma differiscono nel numero di neutroni $N$. Un esempio sono il carbonio-12 ${}^{12} \text{C}$ e il carbonio-14 ${}^{14} \text{C}$, entrambi isotopi del carbonio.
        \item \textbf{Isotoni}: Due nuclidi con uguale $N$ e diverso $Z$. Gli \textit{isotoni} sono dunque atomi che hanno lo stesso numero di neutroni $N$ ma differente numero atomico $Z$. Gli isotoni appartengono quindi ad elementi diversi. Un esempio sono il sodio-23 ${}^{23} \text{Na}$ e il magnesio-24 ${}^{24} \text{Mg}$.
        \item \textbf{Isobari}: Due nuclidi con uguale $A$. Gli \textit{isobari} sono atomi che hanno lo stesso numero di massa, ma differente numero atomico. Gli isobari appartengono quindi ad elementi diversi. Un esempio sono il potassio-40 ${}^{40} \text{K}$ e l'argon-40 ${}^{40} \text{Ar}$.
    \end{itemize}

\section{Unità di misura ed ordini di grandezza}
In fisica nucleare e subnucleare i fenomeni fisici analizzati operano su una scala di dimensioni lineari estremamente piccole che va dai $10^{-15} \, \mathrm{m}$ fino a $10^{-18} \, \mathrm{m}$. Su tali scale usare il metro non è più agibile, per cui si utilizza generalmente un'altra unità di misura di riferimento: il \textbf{fermi}, pari a un \textit{femtometro}.
    \begin{equation*}
        \text{1 fermi} \equiv 1 \, \mathrm{fm} = 10^{-15} \, \mathrm{m}
    \end{equation*}
Le dimensioni dei nuclei variano da circa $1 \, \mathrm{fm}$ per un singolo nucleone a circa $7 \, \mathrm{fm}$ per i nuclei più pesanti. Le masse dei nucleoni e dell'elettrone, pur variando su diversi ordini di grandezza, sono estremamente piccole rispetto al kg. Per ovviare a ciò le masse nucleari sono misurate in termini di \textbf{unità di massa atomica} unificata ($\mathrm{u}$), definita in modo che la massa di un atomo di ${}^{12} \text{C}$ sia esattamente $12 \, \mathrm{u}$.
    \begin{equation*}
        1 \, \mathrm{u} = \frac{m({}^{12} \text{C})}{12}
    \end{equation*}
Quindi i nucleoni hanno masse di circa $1 \, \mathrm{u}$. Nell'analisi dei decadimenti e delle reazioni nucleari, generalmente si lavora con \textbf{energie di massa} piuttosto che con le masse stesse. Per comprendere il concetto di energie di massa ricordiamo che $1 \, \mathrm{eV}$ è l'energia acquisita (o persa) da una singola unità di carica elettrica $e$, che si muove nel vuoto tra due punti dello spazio con una d.d.p. di $1 \, \mathrm{V}$. Essendo un'unità di misura dell'energia, essa può essere convertita in \textit{joule}, dove
    \begin{align*}
        1 \, \mathrm{eV} & = 1.602 \times 10^{-19} \, \mathrm{J} \\
        1 \, \mathrm{J} & = 6,242 \times 10^{18} \, \mathrm{eV} 
    \end{align*}
Questa unità di misura risulta molto utile dato che le energie in gioco quando si studia un sistema nucleare o una reazione nucleare o un processo che coinvolga particelle subnucleari, pur variando su svariati ordini di grandezza, sono estremamente piccole rispetto al \textit{joule}. A titolo di esempio:
    \begin{itemize}
        \item L'energia di legame media per nucleone di un nucleo medio-pesante è dell'ordine di $10^{-12} \, \mathrm{J}$.
        \item La massima energia cinetica impartita fin'ora ad un elettrone con un acceleratore è dell'ordine di $10^{-8} \, \mathrm{J}$
    \end{itemize}
La conversione della massa in energia viene effettuata utilizzando il risultato fondamentale della relatività ristretta $E = mc^2$. In questo modo   
    \begin{equation*}
        1 \, \mathrm{kg} = 5.610 \times 10^{35} \,\mathrm{eV} / \mathrm{c}^2
    \end{equation*}
Per quanto riguarda l'unità di massa atomica il fattore di conversione è $1 \, \mathrm{u} = 931.502 \, \mathrm{MeV}/\mathrm{c}^2$. Così possiamo valutare la massa
dell'elettrone, del protone e del neutrone come
    \begin{table}[ht]
        \centering
        \begin{tabular}{cc}
            \toprule
            Particella & Massa ($\mathrm{MeV}/\mathrm{c}^2$) \\
            \midrule
            Elettrone & $0.5109989$ \\
            Protone & $938.2720$ \\
            Neutrone & $939.5654$ \\
            \bottomrule
        \end{tabular}
        \caption{Masse dell'elettrone e dei nucleoni espresse in $\mathrm{MeV}/\mathrm{c}^2$.}
        \label{tab: particle masses in MeV}
    \end{table}
\subsection{Unità naturali}
Le due costanti fondamentali della meccanica quantistica relativistica sono la costante di Planck $h$, e la velocità della luce nel vuoto $c$:
    \begin{align*}
        \hbar & = \frac{h}{2 \pi} = 1.055 \times 10^{-34} \, \mathrm{J} \cdot \mathrm{s} \\
        c & = 299 \, 792 \, 458 \, \mathrm{m/s}
    \end{align*}
Risulta conveniente utilizzare un sistema di unità in cui $\hbar$ è una unità di azione e $c$ è una unità di velocità. Il nostro sistema di unità sarà completamente definito se ora specificassimo, ad esempio, la nostra unità di energia. Per quanto visto prima risulterà abbastanza comune misurare le quantità in unità di GeV, una scelta motivata dal fatto che l'energia di riposo del protone è approssimativamente di 1\,GeV.

Scegliendo unità con $\hbar = c = 1$, diventa superfluo scrivere esplicitamente $\hbar$ e $\mathrm{c}$ nelle formule, risparmiando così tempo e problemi. Possiamo sempre utilizzare l'analisi dimensionale per determinare inequivocabilmente dove entrano $\hbar$ e $c$ in qualsiasi formula. Pertanto, con una leggera ma consentita pigrizia, è consuetudine parlare di massa, momento, ed energia tutte in termini di GeV, e misurare lunghezza e tempo in unità di $\mathrm{GeV}^{-1}$. La Tabella \ref{tab: natural units} mostra il collegamento tra le unità GeV e le unità mks.
    \begin{table}[ht]
        \centering
        \begin{tabular}{lc}
            \toprule 
            Fattore di conversione & Dimensione effettiva \\ 
            \midrule \vspace{0.2cm}
            $1 \, \mathrm{kg} = 5.61 \times 10^{26} \, \mathrm{GeV}$ & $\displaystyle \frac{\mathrm{GeV}}{\mathrm{c}^2}$ \\ \vspace{0.2cm}
            $1 \, \mathrm{m} = 5.07 \times 10^{15} \, \mathrm{GeV}^{-1}$ & $\displaystyle \frac{\hbar \mathrm{c}}{\mathrm{GeV}}$ \\
            $1 \, \mathrm{sec} = 1.52 \times 10^{24} \, \mathrm{GeV}^{-1}$ & $\displaystyle \frac{\hbar}{\mathrm{GeV}}$ \\
            \bottomrule
        \end{tabular}
        \caption{Unità di massa, lunghezza e tempo convenzionali in termini di unità di energia con $\hbar = c = 1$.}
        \label{tab: natural units}
    \end{table}